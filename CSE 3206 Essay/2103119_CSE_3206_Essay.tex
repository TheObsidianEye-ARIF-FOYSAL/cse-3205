\documentclass[12pt,a4paper]{article}

% ------------------------------------------------
% Packages
% ------------------------------------------------
\usepackage[utf8]{inputenc}
\usepackage{geometry}
\usepackage{graphicx}
\usepackage{titlesec}
\usepackage{xcolor}
\usepackage{fancyhdr}
\usepackage{setspace}
\usepackage{tcolorbox}
\usepackage{hyperref}
\usepackage{enumitem}
\usepackage{booktabs}
\usepackage{ragged2e}

\geometry{margin=1in}
\onehalfspacing

\hypersetup{
    colorlinks=true,
    linkcolor=blue,
    urlcolor=blue
}

% ------------------------------------------------
% Colors
% ------------------------------------------------
\definecolor{ruetblue}{RGB}{25,55,109}
\definecolor{accent}{RGB}{102,0,204}
\definecolor{lecturerorange}{RGB}{255,102,0}
\definecolor{lightgray}{RGB}{245,245,245}

% ------------------------------------------------
% Section Formatting
% ------------------------------------------------
\titleformat{\section}
{\color{ruetblue}\normalfont\Large\bfseries}
{\thesection}{1em}{}

\titleformat{\subsection}
{\color{accent}\normalfont\large\bfseries}
{\thesubsection}{1em}{}

% ------------------------------------------------
% Header/Footer
% ------------------------------------------------
\pagestyle{fancy}
\fancyhf{}
\fancyhead[L]{CSE 3206 -- Software Engineering Sessional}
\fancyhead[R]{CO4 Reflection Essay}
\fancyfoot[C]{\thepage}

% ============================================================
% DOCUMENT START
% ============================================================

\begin{document}

% ============================================================
% SINGLE-PAGE COVER
% ============================================================

\begin{titlepage}
\centering

\vspace*{0.5cm}

{\LARGE \textbf{\textcolor{ruetblue}{RAJSHAHI UNIVERSITY OF ENGINEERING \& TECHNOLOGY}}}\\[0.3cm]
{\large Department of Computer Science \& Engineering}\\[0.6cm]

\textcolor{accent}{\rule{\linewidth}{1.5pt}}\\[0.6cm]

{\LARGE \textbf{\textcolor{accent}{CO4 Reflection Essay}}}\\[0.2cm]
{\large CSE 3206 -- Software Engineering Sessional}\\[0.6cm]

\textcolor{accent}{\rule{\linewidth}{1.5pt}}\\[0.8cm]

% -----------------------------
% Student Info
% -----------------------------

\begin{tcolorbox}[colback=lightgray,colframe=ruetblue,width=0.85\linewidth]
\small
\textbf{Project Title:} Animal Classifier AI\\
\textbf{Submitted by:} Arif Foysal Bin Haider\\
\textbf{Section:} B \quad
\textbf{Series:} 21\\
\textbf{Course Code:} CSE 3206\\
\textbf{Date of Submission:} 12 February, 2026
\end{tcolorbox}

\vspace{0.6cm}

% -----------------------------
% Submitted To
% -----------------------------

\begin{flushleft}
\large \textbf{Submitted to}\\[0.2cm]
\Large \textbf{\textcolor{ruetblue}{Md. Sozib Hossain}}\\
\textcolor{lecturerorange}{Lecturer}\\
Department of Computer Science \& Engineering, RUET
\end{flushleft}

\vspace{0.6cm}

% -----------------------------
% GitHub Link
% -----------------------------

\begin{tcolorbox}[colback=white,colframe=accent,width=0.85\linewidth]
\centering
\small
\textbf{Source Code Repository:}\\
\href{https://github.com/TheObsidianEye-ARIF-FOYSAL/cse-3205/blob/main/CSE%203206%20Essay/2103119_CSE_3206_Essay.tex}
{\textbf{View LaTeX Source on GitHub}}
\end{tcolorbox}

\vfill

\textcolor{ruetblue}{\rule{\linewidth}{1pt}}\\
\small \textit{Heaven's Light is Our Guide}

\end{titlepage}

% ============================================================
% ESSAY CONTENT
% ============================================================

\section*{Reflection Essay: Self-Directed Learning in Software Engineering}
\addcontentsline{toc}{section}{Reflection Essay}

\section{Introduction}

During the CSE 3206 Software Engineering Sessional, I developed a full-stack AI-powered web application titled \textbf{Animal Classifier AI}. The system classifies images of ten animal categories using deep learning and delivers real-time predictions through a web interface. 

While classroom instruction introduced general software engineering principles, this project required me to independently learn and integrate advanced technologies including transfer learning, RESTful API architecture, cloud deployment strategies, and software design patterns.

The most significant technical decision in this project was adopting \textbf{Transfer Learning using MobileNetV2} within TensorFlow/Keras, combined with Flask-based backend architecture. This essay critically reflects on that decision, the challenges of self-directed learning, and how this experience shaped my readiness for lifelong professional growth.

% ------------------------------------------------

\section{Technical Decision and Justification}

\subsection{Chosen Technique: Transfer Learning with MobileNetV2}

The primary new technique I independently learned was \textbf{Transfer Learning using MobileNetV2}. Instead of building a convolutional neural network from scratch, I utilized a pre-trained ImageNet model and added a custom classification head for ten animal classes.

MobileNetV2 was appropriate because:

\begin{itemize}
    \item Lightweight architecture (~3.5 million parameters)
    \item Small model size (10.42MB)
    \item Fast inference time (50--100ms)
    \item High validation accuracy (~95\%)
    \item Optimized for real-time web deployment
\end{itemize}

Its inverted residual blocks and depthwise separable convolutions significantly reduced computational cost while maintaining strong performance.

\subsection{Comparison with Alternative: Custom CNN}

I considered designing a custom CNN from scratch. However:

\begin{itemize}
    \item It required a larger dataset.
    \item It increased overfitting risk.
    \item It produced larger model sizes.
    \item Training time would be longer.
\end{itemize}

\begin{center}
\begin{tabular}{lcc}
\toprule
\textbf{Aspect} & \textbf{Custom CNN} & \textbf{MobileNetV2} \\
\midrule
Parameters & 10M+ & 3.5M \\
Model Size & Large & 10.42MB \\
Inference Speed & Moderate & 50--100ms \\
Deployment Ready & Difficult & Optimized \\
\bottomrule
\end{tabular}
\end{center}

Therefore, MobileNetV2 was technically superior in scalability, memory efficiency, and deployment readiness.

\subsection{Supporting Architectural Decisions}

To improve performance and maintainability, I implemented:

\begin{itemize}
    \item \textbf{Singleton Pattern} for loading the ML model once at startup.
    \item \textbf{Factory Pattern} for standardized image preprocessing.
    \item \textbf{RESTful API architecture} using Flask.
    \item \textbf{Layered MVC structure}.
\end{itemize}

These decisions ensured clean separation of concerns and reduced latency significantly.

% ------------------------------------------------

\section{Self-Directed Learning Experience}

\subsection{Key Challenges}

\textbf{Understanding Transfer Learning:}  
Determining when to freeze layers, when to fine-tune, and selecting appropriate learning rates required deep study of TensorFlow documentation and experimentation.

\textbf{Model Optimization:}  
Initially, the model was large and slow. Through research and iterative testing, I reduced its size to 10.42MB while maintaining high accuracy.

\textbf{Backend Integration:}  
Integrating the ML model into a production-ready Flask server required careful handling of file validation, exception management, and memory optimization.

\textbf{Deployment:}  
Deploying the application using Gunicorn and cloud platforms introduced DevOps practices such as environment configuration and performance monitoring.

\subsection{Evaluation of Learning Resources}

\begin{itemize}
    \item Official TensorFlow documentation: technically comprehensive.
    \item Real Python tutorials: highly practical.
    \item Stack Overflow: useful for debugging.
    \item YouTube tutorials: helpful for conceptual clarity.
\end{itemize}

This process strengthened my ability to critically evaluate and combine multiple learning resources.

% ------------------------------------------------

\section{Reflection and Future Learning Plan}

\subsection{Transformation in Learning Approach}

This project transformed my mindset:

\begin{itemize}
    \item From passive tutorial-following to problem-driven engineering.
    \item From isolated coding tasks to integrated systems thinking.
    \item From hesitation toward complexity to confidence in independent exploration.
\end{itemize}

\subsection{Future Learning Plan: Docker and CI/CD}

My next goal is mastering \textbf{Docker containerization and CI/CD pipelines}.

\textbf{Plan:}
\begin{enumerate}
    \item Learn Docker fundamentals (Weeks 1--2).
    \item Containerize the Animal Classifier AI (Weeks 3--4).
    \item Implement GitHub Actions for automated testing and deployment (Weeks 5--6).
    \item Document the workflow in a technical blog (Week 7).
\end{enumerate}

% ------------------------------------------------

\section{Conclusion}

The Animal Classifier AI project was a transformative experience in self-directed learning. By independently mastering transfer learning, backend architecture, deployment strategies, and software design patterns, I strengthened both technical competence and adaptive learning ability.

In a rapidly evolving technological landscape, continuous learning is essential. This project enhanced my ability to independently evaluate, adopt, and integrate emerging technologies, preparing me for lifelong professional growth as a software engineer.

\vspace{0.5cm}
\textcolor{ruetblue}{\rule{\linewidth}{1pt}}

\end{document}
